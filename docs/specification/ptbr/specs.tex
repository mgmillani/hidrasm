\documentclass[a4paper,10pt]{article}
\usepackage{hyperref}
\usepackage{listings}
\usepackage[T1]{fontenc}
\usepackage[utf8]{inputenc}

\hypersetup
{
	colorlinks=true,
	linkcolor=blue
}

\lstset{
	extendedchars=true,
	inputencoding=utf8,
	basicstyle=\small,
	keywordstyle=\color{red}\bfseries,
	%identifierstyle=\color{blue},
	commentstyle=\color{gray}\textit,
	stringstyle=\color{magenta},
	showstringspaces=false,
	%numbers=left,         % Números na margem da esquerda
	%numberstyle=\tiny,    % Números pequenos
	%stepnumber=1,         % De 1 em 1
	%numbersep=5pt,         % Separados por 5pt
	language=[LaTeX]TeX
}

\lstset
{
	literate={ê}{{\^e}}1 {ç}{{\c{c}}}1 {ã}{{\~a}}1 {â}{{\^a}}1 {õ}{{\~o}}1 {í}{{\'i}}1 {é}{{\'e}}1 {á}{{\'a}}1 {ó}{{\'o}}1 {ú}{{\'u}}1
}

\newcommand{\Enderecamentos} {\hyperref[endereçamentos]{Endereçamentos}}
\newcommand{\Instrucoes} {\hyperref[instruções]{Instruções}}
\newcommand{\Maquina} {\hyperref[máquina] {Máquina}}
\newcommand{\Registradores} {\hyperref[registradores] {Registradores}}

%opening
\title{Especificação do Hidrasm}
\author{PET Computação da UFRGS}

\begin{document}

\maketitle

\tableofcontents

\section{Base}
Para simplicidade, o programa do montador possui somente uma interface por linha de comando. Para integração com o simulador, a interface chama esse programa. Isso é feito para, caso deseje-se mudar o montador, não será necessário alterar o código fonte da interface ou do simulador.

Cada máquina é especificada em um arquivo, o qual contém as seguintes seções:
\begin{itemize}
 \item \Enderecamentos
 \item \Instrucoes
 \item \Maquina
 \item \Registradores
\end{itemize}

\section{Arquivo de Descrição}

	\subsection{Endereçamentos}
		\label{endereçamentos}
		Descreve os modos de endereçamento de cada máquina, como o formato pelo qual é identificado, sobre quais tipos é válido e se é relativo ao PC ou não.
		
		\begin{description}
		 \item [Identificador] addressings
		 \item [Formato] \verb+<nome> [= | -] <código binário> <identificador> +
		\end{description}
		onde:
		\begin{description}
		 \item [nome] nome do modo de endereçamento (usado internamente). Pode ser qualquer sequência de caracteres a-z e A-Z;
		 \item [= | -] Determina se o valor do endereço deve ser calculado em relação ao PC ou não. "=" indica que deve ser simplesmente copiado, "-" que é relativo ao PC. (Padrão: =);
		 \item [código binário] número;
		 \item [identificador] Expressão que identifica o modo de endereçamento. Somente um registrador ou endereço pode aparecer por expressão.
		\end{description}

		\subsection{Instruções}
		\label{instruções}
		Descreve qual o mnemônico, formato, modos de endereçamento e registradores possíveis para cada instrução.
		
		\begin{description}
		 \item [Identificador] instructions
		 \item [Formato] <tamanho> <mnemônico> <operandos> <endereçamento0, \dots> < <registrador0, \dots> | - | *> <formato binário>
		\end{description}
		onde:
		\begin{description}
		 \item [tamanho] número de bits da instrução;
		 \item [mnemônico] qualquer sequência dos caracteres a-z e A-Z (case insensitive);
		 \item [operandos] Expressão que indica como os operandos da instrução devem aparecer. Uma expressão sem variáveis será ignorada e indicará que nenhum operando deve ser utilizado;
		 \item [endereçamento0 \dots] quais modos de endereçamento podem ser usados com essa instrução. * indica qualquer um;
		 \item [formato binário] como que a instrução é montada. São usados os símbolos:
			\begin{description}
			 \item [a{[[n]][(m)]}] n-ésimo endereço (se o n nâo for informado, são colocados na ordem em que aparecem). m indica quantos bits terá o endereço, sendo desnecessário quando há somente um endereço no formato. Se houver mais de um endereço na expressão, todos usarão o mesmo número de bits;
			 \item [m{[n]}] n-ésimo modo de endereçamento (se o n nâo for informado, são colocados na ordem em que aparecem);
			 \item [r{[n}]  n-ésimo registrador (se o n nâo for informado, são colocados na ordem em que aparecem);
			 \item [0] bit 0;
			 \item [1] bit 1;
			 \item [qualquer outro caractere] ignorado
			\end{description}
		\end{description}
		
		\subsubsection{Exemplo}
			\verb+16 ADD a d - 0011000ma[0](8)+

	\subsection{Máquina}
		\label{máquina}
		Descrever diversos aspectos da máquina, como tamanho do PC e a endianness usada.
		\begin{description}
		 \item [Identificador] machine
		 \item [Formato] <little-endian | big-endian> <bits do PC>
		\end{description}
		onde:
		\begin{description}
		 \item [little-endian | big-endian] endianness usada;
		 \item [bits do pc] quantos bits o PC possui (determina o tamanho máximo da memória).
		\end{description}

		\subsubsection{Exemplo}
			little-endian 8
			
	\subsection{Registradores}
		\label{registradores}
		
		Descreve cada registrador usado pela máquina que pode aparecer no código assembly, definindo seu identificador.
		\begin{description}
		 \item [Identificador] registers
		 \item [Formato] <nome> <valor>
		\end{description}
		onde:
		\begin{description}
		 \item [nome] nome do registrador (para o código Assembly).
		 \item [valor] número que indica o valor do registrador (para o código de máquina)
		\end{description}

		\subsubsection{Exemplo}
			R0 0b
			
\section{Expressão}
\label{expressão}
Uma expressão identifica um padrão de forma similar a uma expressão regular. São usados os símbolos: 

\begin{description}
	\item [r] registrador 
	\item [n] endereço numérico 
	\item [l] label 
	\item [a] n ou l 
	\item [o] qualquer um dos acima (o de operando) 
	\item [\textbackslash] caractere de escape 
	\item [] qualquer outro caractere indica algum que deve aparecer na frase
\end{description}

As variáveis identificadas podem ter somente os caracteres em a-zA-Z0-9 e \_ . Elas devem estar separadas por um caractere em branco se a expressão exigir algum dos caracteres que as compõem (ver exemplo). 

Ao ler uma expressão, espaços entre os caracteres são ignorados. Por isso, uma expressão não pode conter espaços (se contiver, estes serão ignorados).

	\subsection{Exemplo}
	As seguintes frases satisfazem "(as),a":
	\begin{itemize}
		\item "(125 s),red" (match de 125 e red) 
		\item " ( reds s) , 125" (match de reds e 125)
	\end{itemize}
	
	As seguintes frases não satisfazem "(as),a": 
	\begin{itemize}
		\item "1(24 s),red" 
		\item "(reds),32" 
		\item "((red s),32"
	\end{itemize}

\section{Números}
São suportados números nas seguintes bases:
\begin{description}
	\item [binário] seguido de um "b" ou "B" 
	\item [decimal] seguido de um "d" ou "D" ou nada 
	\item [hexadecimal] precedido por um algarismo e seguido de um "h" ou "H". Os caracteres dos número podem ser minúsculos ou maiúsculos.
	
	\subsection{Exemplos}
	\begin{description}
	 \item [010b] 2
	 \item [010] 10
	 \item [010d] 10
	 \item [010h] 16
	 \item [ah] Não corresponde a um número válido
	\end{description}

\end{description}


			
\section{Erros e Avisos}

As mensagens de erro e os avisos estarão em um arquivo separado, de forma a facilitar eventuais traduções do programa além de permitir diferentes níveis de rigidez para os erros.

	\subsection{Arquivo}
		\begin{description}
			\item [Formato] <código> <e | w> <mensagem>
		\end{description}
		onde:
		\begin{description}
			\item [código] número da mensagem
			\item [e] erro
			\item [w] aviso
			\item [mensagem] a mensagem que será usada.
		\end{description}
		
		Comentários são feitos com o símbolo '\#'.
		
		\subsubsection{Variáveis}
			Para que o usuário possua mais informações, as mensagens podem conter variáveis que descrevem a linha em que ocorreu o problema. São elas:
			
			\begin{description}
				\item [ADDRESSING\_MODE] modo de endereçamento usado; 
				\item [DISTANCE] valor, em bytes, do último endereço relativo ao PC (sem truncamentos);
				\item [EXPECTED\_OPERANDS] número de operandos esperado pela diretiva ou instrução;
				\item [FOUND\_OPERANDS] número de operandos encontrados;
				\item [LABEL] nome da label;
				\item [LAST\_ORG\_LINE] número da linha do último ORG encontrado;
				\item [LINE] a linha atual do código;
				\item [MNEMONIC] mnemônico da instrução ou diretiva;
				\item [OPERAND\_SIZE] valor máximo que o operando pode assumir.
			\end{description}
	
	\subsubsection{Exemplo}
	\begin{lstlisting}
#Erros 

(1) 0 e Instrução ou diretiva desconhecida: $MNEMONIC 
(2) 1 e Número incorreto de operandos: esperava-se 
$EXPECTED_OPERANDS, encontrou-se $FOUND_OPERANDS 
(3) 2 e Endereço relativo ao PC muito grande: 
$LABEL está a uma distância de $DISTANCE bytes do
PC, mas o máximo possível são $OPERAND_SIZE bytes
(4) #Avisos
(5)
(6) 3 w Região possivelmente sobrescrita a partir
da linha $LAST_ORG_LINE
(7) 4 w Modo de endereçamento desconhecido: 
$ADDRESSING_MODE		 
	\end{lstlisting}
	
	onde os números entre {()} indicam as linhas no arquivo

\end{document}
